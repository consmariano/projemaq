\chapter{Introdução}

\setcounter{page}{1}


\section{Subgrupos}

Para obter uma otimização na organização do projeto, e garantir uma divisão justa e igual para todos os membros, o grupo se dividiu em subgrupos, cada um responsável por uma parte importante do projeto. É importante ressaltar que os membros da equipe são responsáveis por ajudar em todas as partes do trabalho, não somente o subgrupo em que está inserido, de forma que seja possível uma boa comunicação entre os times. A divisão dos subgrupos escolhida foi a seguinte:

\begin{itemize}
    \item \textbf{Gerência:} a gerente do grupo é responsável por garantir uma boa organização do trabalho como um todo, além de verificar se todos os membros do grupo estão participando do projeto.
    \item \textbf{Mecânica:} este subgrupo tem como função garantir o bom funcionamento da parte estrutural da máquina: desde o esboço inicial, passando pela escolha de materiais, até a montagem do protótipo.
    \item \textbf{Eletrônica:} esta equipe cuida de todos os componentes elétricos do torno, suas conexões e seu bom funcionamento.
    \item \textbf{Programação:} Este time deve realizar a parte da lógica de funcionamento e do código Linux CNC da máquina.
\end{itemize}

É notável que todos os subgrupos devem se comunicar entre si para o bom funcionamento do projeto, e é claro, trabalhar em conjunto para que a máquina funcione da maneira desejada.

